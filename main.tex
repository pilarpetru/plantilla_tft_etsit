\documentclass[a4paper,12pt]{book}
\usepackage[utf8]{inputenc}
\usepackage[spanish]{babel}
\usepackage{fancyhdr}
\usepackage{amsmath, mathtools}
\usepackage{amssymb}
\usepackage{eurosym}    
\usepackage{graphicx}
\usepackage{float} 
\usepackage{vmargin}
\usepackage{cancel}
\usepackage{caption}
\usepackage{subcaption}
\usepackage{appendix}   
\usepackage[pdftex]{hyperref}   
\hypersetup{colorlinks=true,linkcolor=black,linkcolor=black,filecolor=black,urlcolor=black, citecolor=black}    \urlstyle{same} %hipervinculos en negro y sin subrayar
\usepackage[nottoc,notlot,notlof]{tocbibind}
\usepackage{listings}   
\usepackage{xcolor} 
\usepackage{wrapfig}    
\usepackage{transparent}
\usepackage{import}
\usepackage{verbatim}
\usepackage{afterpage}
\usepackage{emptypage}
\usepackage{etoolbox}
\usepackage{parskip}
\usepackage{amssymb}

\makeatletter
\definecolor{nar}{HTML}{FF8000}
\renewcommand\mainmatter{
    \clearpage
  \@mainmattertrue
  \pagenumbering{arabic}}

\renewcommand\lstlistingname{Programación}
\renewcommand\lstlistlistingname{Índice de programación}
\addto\captionsspanish{\renewcommand*\contentsname{Índice}}
\addto\captionsspanish{\renewcommand{\tablename}{Tabla}  
          \renewcommand{\listtablename}{Índice de tablas}}

% cambiar márgenes chapters
\patchcmd{\@makechapterhead}{50\p@}{\chapheadtopskip}{}{}% Space from top of page to CHAPTER X
\patchcmd{\@makechapterhead}{20\p@}{\chapheadsep}{}{}% Space between CHAPTER X and CHAPTER TITLE
\patchcmd{\@makechapterhead}{40\p@}{\chapheadbelowskip}{}{}% Space between CHAPTER TITLE and text
% --- Patch \chapter*
\patchcmd{\@makeschapterhead}{50\p@}{\chapheadtopskip}{}{}% Space from top of page to CHAPTER TITLE
\patchcmd{\@makeschapterhead}{40\p@}{\chapheadbelowskip}{}{}% SPace between CHAPTER TITLE and text
\makeatother
% Set new lengths
\newlength{\chapheadtopskip}\setlength{\chapheadtopskip}{20pt}
\newlength{\chapheadsep}\setlength{\chapheadsep}{40pt}
\newlength{\chapheadbelowskip}\setlength{\chapheadbelowskip}{15pt}

\renewcommand{\arraystretch}{1.3}

%Dimensiones
\textwidth = 16cm
\textheight = 25cm
\topmargin = 1cm
\oddsidemargin = 2.5cm
\evensidemargin = 2.5cm
\headsep = 1cm
\footskip = 1.5cm

%Encabezado y pie de página

\makeatletter
\renewcommand\chapter{\if@openright\cleardoublepage\else\clearpage\fi
\thispagestyle{fancy}% original style: plain
\global\@topnum\z@
\@afterindentfalse
\secdef\@chapter\@schapter}
%la siguiente línea es para que no haya páginas en blanco entre los capítulos
\patchcmd{\chapter}{\if@openright\cleardoublepage\else\clearpage\fi}{}{}{}
%para quitar lo de "capítulo 1"
\def\@makechapterhead#1{%
  \vspace*{50\p@}%
  {\parindent \z@ \raggedright \normalfont
    \ifnum \c@secnumdepth >\m@ne
      \if@mainmatter
        %\huge\bfseries \@chapapp\space \thechapter
        \Huge\bfseries \thechapter.\space%
        %\par\nobreak
        %\vskip 20\p@
      \fi
    \fi
    \interlinepenalty\@M
    \Huge \bfseries #1\par\nobreak
    \vskip 40\p@
  }}
\makeatother %para usar lo mismo en la primera pag del capitulo


\pagestyle{fancy}
\fancyhf{}
\fancyhead[LE,RO]{\transparent{0.6}\includegraphics[width=3cm]{portada/logosportada/logoetsit.png}}
\fancyhead[RE,LO]{\transparent{0.6}\thepage}
\chead{\transparent{0.6}Nombre del autor}
\fancyfoot[CE,CO]{\transparent{0.6}Título completo del TFG}
\renewcommand{\headrulewidth}{0.4pt} % grosor de la línea de la cabecera
\renewcommand{\footrulewidth}{0.4pt} % grosor de la línea del pie


\begin{document}

%Numeracion romana
\frontmatter

%Portada
\begin{titlepage}
\pagecolor{nar}
\afterpage{\nopagecolor}
    \begin{center}
        \vspace{1cm}
        \LARGE{\textsc{Universidad Politécnica de Madrid}}\par
        \vspace{0.3cm}
        \large{\textsc{Escuela Técnica Superior de Ingenieros de Telecomunicación}} \par
    \end{center}
    
    \begin{figure}[H]
        \vspace{1cm}
        \centering
        \includegraphics[width=10cm]{portada/logosportada/byn.png}
    \end{figure}

    \begin{center}
        \vspace{1.5cm}
        \Large{\textsc{Grado en Ingeniería...}}\par
        \vspace{0.2cm}
        \LARGE{\textsc{Trabajo Fin de Grado}}\par
        \vspace{2.5cm}

        \Huge{\textsc{Título completo del TFG}}

        \vspace*{2.5cm}
                  \large{\textsc{ Nombre del autor }}
        \par\vspace{2cm}
               \small{\today}
    \end{center}
\end{titlepage}
\empty

% Segunda portada
\begin{titlepage}
\Large{\textsc{\textcolor{nar}{Grado en XXXXXXXXXXXXXXXXX}}} \par
\vspace{0.3cm}
\colorbox{nar}{\Large{\textsc{Trabajo Fin de Grado}}} \par
\vspace{0.2cm}
\normalsize{\textbf{Título:} ……………….} \par
\vspace{0.2cm}
\normalsize{\textbf{Autor:} D…………………….} \par
\vspace{0.2cm}
\normalsize{\textbf{Tutor:} D. ……………….} \par
\vspace{0.2cm}
\normalsize{\textbf{Ponente:} D. ………………} \par
\vspace{0.2cm}
\normalsize{\textbf{Departamento:} …………} \par
\vspace{1cm}
\colorbox{nar}{\Large{\textsc{Miembros del Tribunal}}} \par
\vspace{0.2cm}
\normalsize{\textbf{Presidente:} D. ……………} \par
\vspace{0.2cm}
\normalsize{\textbf{Vocal:} D. …………..} \par
\vspace{0.2cm}
\normalsize{\textbf{Secretario:} D. …………..} \par
\vspace{0.2cm}
\normalsize{\textbf{Suplente:} D. ……………..} \par
\vspace{0.8cm}
\normalsize{\quad \quad \quad       Los miembros del tribunal arriba nombrados acuerdan otorgar la calificación de: ………} \par
\vspace{2cm}
\flushright
\normalsize{Madrid, a \quad \quad\quad de \quad \quad\quad\quad\quad\quad\quad de 20...} \par
\end{titlepage}
\empty
\input{portada/portadaBlanca.tex}
\empty

% Resumen y agradecimientos 
\chapter*{}
\addcontentsline{toc}{chapter}{Resumen y Palabras Clave}
% Resumen en español
\Large{\textsc{\textcolor{nar}{Resumen}}} \par
\vspace{0.2cm}
\normalsize{Número máximo de palabras: 500

...}

\vspace{0.9cm}

% Summary (resumen en inglés)
\Large{\textsc{\textcolor{nar}{Summary}}} \par
\vspace{0.2cm}
\normalsize{Maximum number of words: 500

...}
\vspace{0.9cm}

% Palabras clave
\Large{\textsc{\textcolor{nar}{Palabras Clave}}} \par \par
\vspace{0.2cm}
\normalsize{Deben reflejar el contenido del trabajo, deberían servir para localizar el TFG mediante búsqueda bibliográfica...} \par
\vspace{0.8cm}

\Large{\textsc{\textcolor{nar}{Keywords}}} \par
\vspace{0.2cm}
\normalsize{...} \par

\newpage
\input{pre/greetings}

% Indice, lista de figuras y de tablas
\newpage
\begin{center}
   \tableofcontents
\end{center}
\newpage
\begin{center}
   \listoffigures
   \listoftables
\end{center}
\newpage

% Lista de acrónimos
    \chapter{Lista de acrónimos}
\paragraph{Acrónimo} \textit{Significado...}
\newpage

\mainmatter

\chapter{Introducción y objetivos}
    \import{capitulos/cap1/}{intro.tex}
    \import{capitulos/cap1/}{objetivos.tex}
\newpage

\chapter{Marco teórico}
    \import{capitulos/cap2/}{ejemplo.tex}
\newpage

\chapter{Desarrollo}
    \import{capitulos/cap3/}{ejemploDesarrollo.tex}
\newpage

\chapter{Resultados}
    \import{capitulos/cap4/}{ejemploResultados.tex}
\newpage

\chapter{Conclusiones y líneas futuras}
    \import{capitulos/cap5/}{conclusiones.tex}
    \import{capitulos/cap5/}{lineasfuturas.tex}

\newpage
\bibliographystyle{unsrt}
\bibliography{biblio}

\backmatter
\chapter{Anexo A: Aspectos éticos, económicos, sociales y ambientales}
    \import{capitulos/anexos/}{anexoA.tex}


\newpage
\chapter{Anexo B: Presupuesto económico}
    \import{capitulos/anexos/}{anexoB.tex}
\newpage



\end{document}