\normalsize{El apartado ``Requisitos de las acreditaciones internacionales EUR-ACE y ABET'' de la Normativa de TFT de la ETSIT-UPM establece que ``\textit{La memoria del TFT del GITST, GIB y MUIT, y en general la de aquellas titulaciones que hayan obtenido o para las que se desee solicitar una acreditación internacional EUR-ACE o ABET, debe mostrar conciencia de la responsabilidad de la aplicación práctica de la ingeniería, el impacto social y ambiental, el compromiso con la ética profesional, la responsabilidad y las normas de la aplicación práctica de la ingeniería, así como sobre las prácticas de gestión de proyectos, gestión, y control de riesgos, entendiendo sus limitaciones}''.

Este anexo obligatorio del TFT tendrá un carácter sintético con los siguientes apartados:}
\vspace{0.1cm}

\Large{\textsc{\textcolor{nar}{A1. Introducción}}} \par
\normalsize{Breve descripción del contexto del proyecto, objetivos, necesidades que pretende cubrir o problemas que pretende resolver, centrándose en su relación con los temas sociales, económicos, éticos, legales y/o ambientales que se hayan identificado.}
\vspace{0.1cm}

\Large{\textsc{\textcolor{nar}{A2. Descripción de impactos relevantes relacionados con el proyecto}}} \par
\vspace{0.3cm}
\normalsize{Síntesis del trabajo realizado en la fase 3, de selección y descripción de impactos. Presentar y justificar las conclusiones a las que se haya llegado sobre cuáles son los asuntos más relevantes relacionados con la sostenibilidad social, económica o ambiental, así como los principales grupos de interés identificados y que se han considerado en los análisis posteriores. }
\vspace{0.1cm}

\Large{\textsc{\textcolor{nar}{A3. Análisis detallado de alguno de los impactos}}} \par
\vspace{0.3cm}
\normalsize{ Síntesis del trabajo de análisis realizado. }


\Large{\textsc{\textcolor{nar}{A4. Conclusiones}}} \par
\vspace{0.3cm}
\normalsize{Valorar el proyecto desde un punto de vista ético, social, económico y medioambiental y justificar si el uso de criterios de sostenibilidad ha aportado o puede aportar valor añadido al proyecto.}